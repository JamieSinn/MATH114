\documentclass[12pt]{article}
\usepackage[margin=1in]{geometry}
\usepackage{amsthm,amssymb,amsfonts}
\usepackage{tipa}
\usepackage{ gensymb }
\usepackage[fleqn]{amsmath}

\newcommand{\N}{\mathbb{N}}
\newcommand{\Z}{\mathbb{Z}}


\newcommand{\dbarl}{\left\lVert}
\newcommand{\dbarr}{\right\rVert}


\newenvironment{problem}[2][Problem]{\begin{trivlist}
\item[\hskip \labelsep {\bfseries #1}\hskip \labelsep {\bfseries #2.}]}{\end{trivlist}}
%If you want to title your bold things something different just make another thing exactly like this but replace "problem" with the name of the thing you want, like theorem or lemma or whatever

\begin{document}

%\renewcommand{\qedsymbol}{\filledbox}
%Good resources for looking up how to do stuff:
%Binary operators: http://www.access2science.com/latex/Binary.html
%General help: http://en.wikibooks.org/wiki/LaTeX/Mathematics
%Or just google stuff

\title{MATH 114, Fall 2016. Assignment 2}
\author{James Sinn, 20654551}
\maketitle

\begin{problem}{1}
	Angle between two vectors
\end{problem}
	\[\cos\theta = \frac{-2}{2*2}\]
	\[\cos\theta = \frac {-1}{2}\]
	\[\arccos{\frac{-1}{2}}\]
	\[90\degree or \frac{\pi}{2}\]
\begin{problem}{2}
	Triangle Inequality
\end{problem}
	\[\dbarl\vec x\dbarr = \sqrt{5}\]
	\[\dbarl\vec y\dbarr = \sqrt{5}\]
	\[\dbarl\vec x\dbarr + \dbarl\vec y\dbarr = 2\sqrt{5}\]
	\[\dbarl\vec x + \vec y \dbarr = 3\sqrt{5}\]
	\[3\sqrt{5} \geq 2\sqrt{5}\]
	\[\vec x \text{ and } \vec y  \text{ both have the same direction vector.}\]
	\[\vec y \text{ is } \vec x \text{ scaled by a factor of 3. Because of this the Triangle Inequality is satisfied.}\]
	\[\dbarl \vec x\dbarr = \sqrt{5}\]
	\[\dbarl \vec y\dbarr = 3\sqrt{5}\]
	\[\dbarl \vec x + \vec y\dbarr = 4\sqrt{5}\]
	\[ 4\sqrt{5} \leq \sqrt{5} + 3\sqrt{5} \]
\begin{problem}{3}
	Work - Dot Product
\end{problem}
	\[ 100\cos{\frac{\pi}{6}} = \vec F \dot \vec d \]
	\[ 100\frac{\sqrt{3}}{2} = 50\sqrt{3}\]
	\[F = 50\sqrt{3}J\] 
\begin{problem}{4}
	Projections
\end{problem}
	The projection on the horizontal is the horizontal component of F and the perpendicular of the horizontal is they vertical component of F.
	To calculate the projection, the formula of $\frac{\vec F \cdot \vec H}{\dbarl \vec H \dbarr}$ can be used to obtain the horizontal component. 
\begin{problem}{5}
	Lines and Projections
\end{problem}
a)\\
	The vector equation of $ y=\frac{-3}{2}x + 2$  is 
	$ \vec v = \left[\begin{matrix} 2\\-3 \end{matrix}\right]t + \left[\begin{matrix} 0\\2 \end{matrix}\right] $
	This does not satisfy the requirements of a subspace of $ \rm I\!R^2 $ because it does not intersect the origin. And thus is not a subspace.\\
b)\\
	\[\vec x = \left[\begin{matrix} 1\\9\end{matrix}\right] \vec y = \left[\begin{matrix} -4\\6\end{matrix}\right]\]

	\[\dbarl \vec x \dbarr = \sqrt{82}\]

	\[\text{proj}_{\vec{x}}\vec y = \frac{\vec x \cdot \vec y}{\dbarl\vec y\dbarr} \]
	\[\text{proj}_{\vec{x}}\vec y = \frac{20}{2\sqrt{13}}\]
	\[\text{proj}_{\vec{x}}\vec y = \frac{10}{\sqrt{13}}\]\\
c)\\
	\[\vec x = \left[\begin{matrix}5\\3\end{matrix}\right] \vec y = \left[\begin{matrix}-4\\6\end{matrix}\right]\]
	\[\text{proj}_{\vec x}\vec y = \frac{\vec x \cdot \vec y}{\dbarl\vec y\dbarr}\]
	\[\text{proj}_{\vec x}\vec y = \frac{-2}{2\sqrt{13}}\]
	\[\text{proj}_{\vec x}\vec y = \frac{-1}{\sqrt{13}}\]
d)\\
e)\\

%	I dislike Projections
\begin{problem}{6}
	Scalar Equation of a Plane
\end{problem}
	\[\vec{ PQ} = \left[\begin{matrix}1\\1\\-4\end{matrix}\right]\]
	\[\vec{ PR} = \left[\begin{matrix}0\\-5\\-2\end{matrix}\right]\]
	\[\vec n = \left[\begin{matrix}(1)(-2) - (-5)(-4)\\(-4)(0) - (1)(-2)\\(1)(-5) - (1)(0)\end{matrix}\right]\]
	\[\vec n = \left[\begin{matrix}-22\\2\\-5\end{matrix}\right]\]
	\[\vec P = -22x + 2y -5z\]
	\[-22(x-1) + 2(y-5) -5(z-3) = 0\]
	\[-22x +2y -5z = 27\]
\begin{problem}{7}
	Cross Product
\end{problem}
a)\\
	\[\vec w = \left[\begin{matrix}(2)(3) - (2)(-1)\\(2)(-2) - (1)(1)\\(1)(-1) - (2)(-2)\\\end{matrix}\right]\]
	\[\vec w = \left[\begin{matrix}	8\\-4\\5\\\end{matrix}\right]\]
	\[\dbarl \vec w \dbarr = \sqrt{105}\]
b)\\
	\[\dbarl \vec x \dbarr = 3\]
	The answer will be the same because $\vec y$ is $\vec x$ flipped. This is the definition of the perpendicular vector.\\
c)\\
	\[\vec a = \left[\begin{matrix}1\\5\\5\end{matrix}\right]\]
	
	\[\text{proj}_{\vec a}\vec b = \frac{\vec a \cdot \vec b}{\dbarl \vec b \dbarr}\]
	\[\vec a \cdot \vec b = \dbarl\vec a \dbarr \dbarl \vec b \dbarr \cos{\frac{pi}{4}}\]
	\[\vec a \cdot \vec b = \sqrt{51} \dbarl \vec b \dbarr (-\frac{1}{4})\]
	
	
	
\begin{problem}{8}
	Cross Product Theory
\end{problem}
a)\\
	When $\vec u \cdot \vec v$ equals 0, the two vectors are orthagonol.
	The case of when the cross product of the two is equal to 0 is when either $\vec u $ or $ \vec v$
	is equal to the $ \vec 0 $ vector.
	Alternatively, but not applicable to this case is when the two vectors are parallel or antiparallel.\\
b)\\
	$\vec w$ is the third and final vector that can be orthagonal to both $\vec u $ and $\vec v$. Unless the dimension is changed, there cannot be another vector that meets this.\\
	The space would need to be $\rm I\!R^4$ or greater in order to solve this question.
\begin{problem}{9}
	Complex Numbers
\end{problem}
a)\\
	\[(1+ i\sqrt{3})(1 + i)\]
	\[(1+i)-(1-i)\sqrt{3}\]
	\[1 - \sqrt{3} + i(1 + \sqrt{3})\]
b)\\
	\[\frac{\pi + i\pi}{1 - \sqrt{3i}}\]
	\[\frac{\pi + i\pi}{1 - \sqrt{3i}} \times \frac{1 + \sqrt{3i}}{1 + \sqrt{3i}} \]
	\[\frac{(1 -i) \pi}{\sqrt{3} + i}\]

\begin{problem}{10}
	Solving Polynomials
\end{problem}
	As we are given a factor, we can use synthetic division to get the quadratic to solve.
	The quadratics specified below are the result of it.\\
a)\\
	\[x^2 -x + 3 = 0\]
	\[x = \frac{-1 \pm \sqrt{1 - 4(1)(3)}}{2}\]
	\[x = \frac{-1 \pm \sqrt{-11}}{2}\]
	Root 1 $x = \frac{1}{2} (-1 + 11i)$\\
	Root 2 $x = \frac{1}{2} (-1 - 11i)$\\
b)\\
	\[x^2 -2x -2 = 0\]
	\[x = \frac{2 \pm \sqrt{4 - 4(1)(-2)}}{2}\]
	\[x = \frac{1}{2} (2 \pm 2\sqrt{3})\]
	Root 2 $x = \frac{1}{2} (2 + 2\sqrt{3})$\\
	Root 3 $x = \frac{1}{2} (2 - 2\sqrt{3})$\\
	
\end{document}
