\documentclass[12pt]{article}
\usepackage[margin=1in]{geometry}
\usepackage{amsthm,amssymb,amsfonts}
\usepackage{tipa}
\usepackage{ gensymb }
\usepackage[fleqn]{amsmath}
\usepackage{dsfont}

\newcommand{\N}{\mathbb{N}}
\newcommand{\Z}{\mathbb{Z}}
\newcommand{\dbarl}{\left\lVert}
\newcommand{\dbarr}{\right\rVert}
\newcommand{\sqbrl}{\left[}
\newcommand{\sqbrr}{\right]}

\newenvironment{problem}[2][Problem]{\begin{trivlist}
\item[\hskip \labelsep {\bfseries #1}\hskip \labelsep {\bfseries #2.}]}{\end{trivlist}}
%If you want to title your bold things something different just make another thing exactly like this but replace "problem" with the name of the thing you want, like theorem or lemma or whatever
 
\begin{document}
 
%\renewcommand{\qedsymbol}{\filledbox}
%Good resources for looking up how to do stuff:
%Binary operators: http://www.access2science.com/latex/Binary.html
%General help: http://en.wikibooks.org/wiki/LaTeX/Mathematics
%Or just google stuff
 
\title{MATH 114 - Fall 2016 - Assignment 3}
\author{James Sinn - 20654551}
\maketitle
 

\begin{problem}{1}
	Complex Numbers - Euler's Formula
\end{problem}
a)\\
	\[= e^{i\frac{\pi}{2}}\]
	\[= \cos{\frac{\pi}{2}} + i \sin{\frac{\pi}{2}}\]
	\[= 0 + i (1)\]
	\[= i\]
b)\\	
	\[= e^{2\pi \frac{i}{3}}\]
	\[= \cos{2\pi} + \frac{i}{3} \sin{2\pi}\]
	\[= 1 + \frac{i}{3} (0)\]
	\[= 1\]
\begin{problem}{2}
	Complex Numbers - Euler's Formula
\end{problem}
	\[z = \cos{\frac{\pi}{3}} + i \sin{\frac{\pi}{3}}\]
	\[z = e^{i\frac{\pi}{3}}\]
	\[z^6 = e^{6i2\pi}\]
	\[= \cos{2\pi} + 6i \sin{2\pi}\]
	\[= 1 + 6i(0)\]
	\[= 1\]\\\\\\
\begin{problem}{3}
	Systems of Equations
\end{problem}
a)\\
	\[= 
		\left[
		\begin{array}{cc|c}
		3 & 2 & -9 \\
		1 & 4 & 7\\
		\end{array}
		\right] R_2 - R_1
	\]
	\[\sim
		\left[
		\begin{array}{cc|c}
		3 & 2 & -9 \\
		-2 & 2 & 16\\
		\end{array}
		\right] R_1 + R_2
	\]
	\[\sim
		\left[
		\begin{array}{cc|c}
		1 & 4 & 7 \\
		-2 & 2 & 16\\
		\end{array}
		\right] \frac{1}{2} R_2
	\]
	\[\sim
		\left[
		\begin{array}{cc|c}
		1 & 4 & 7 \\
		-1 & 1 & 8\\
		\end{array}
		\right] R_2 + R_1
	\]
	\[\sim
		\left[
		\begin{array}{cc|c}
		1 & 4 & 7 \\
		0 & 5 & 15\\
		\end{array}
		\right] \frac{1}{5} R_2
	\]
	\[\sim
		\left[
		\begin{array}{cc|c}
		1 & 4 & 7 \\
		0 & 1 & 3\\
		\end{array}
		\right] R_1 - 4R_2
	\]
	\[\sim
		\left[
		\begin{array}{cc|c}
		1 & 0 & -5 \\
		0 & 1 & 3\\
		\end{array}
		\right]
	\]
	\[\vec x_1 = -5,  \vec x_2 = 3\]
b)\\
	\[= 
		\left[
		\begin{array}{ccc|c}
		1 & 2 & 3 & 1 \\
		1 & 1 & 1 & 2\\
		1 & 3 & 4 & 0\\
		\end{array}
		\right] R_3 - R_1
	\]
	\[\sim 
		\left[
		\begin{array}{ccc|c}
		1 & 2 & 3 & 1 \\
		1 & 1 & 1 & 2\\
		0 & 1 & 1 & -1\\
		\end{array}
		\right] R_2 - R_3
	\]
	\[\sim 
		\left[
		\begin{array}{ccc|c}
		1 & 2 & 3 & 1 \\
		1 & 0 & 0 & 3\\
		1 & 1 & 1 & -1\\
		\end{array}
		\right] R_2 \Leftrightarrow R_1
	\]
	\[\sim 
		\left[
		\begin{array}{ccc|c}
		1 & 0 & 0 & 3 \\
		1 & 2 & 3 & 1\\
		1 & 1 & 1 & -1\\
		\end{array}
		\right] R_2 - 2R_3 - R_1
	\]
	\[\sim
		\left[
		\begin{array}{ccc|c}
		1 & 0 & 0 & 3 \\
		0 & 0 & 1 & 0\\
		0 & 1 & 1 & -1\\
		\end{array}
		\right] R_3 - R_2
	\]
	\[\sim
		\left[
		\begin{array}{ccc|c}
		1 & 0 & 0 & 3 \\
		0 & 0 & 1 & 0\\
		0 & 1 & 0 & -1\\
		\end{array}
		\right] R_3 \Leftrightarrow R_2
	\]
	\[\sim
		\left[
		\begin{array}{ccc|c}
		1 & 0 & 0 & 3 \\
		0 & 1 & 0 & -1\\
		0 & 0 & 1 & 0\\
		\end{array}
		\right]
	\]
Inconsistent, therefore no solutions.\\

c)\\
It is not possible to reduce to RREF fully, the closest that I could get is below.
This is because it is inconsistent.
	\[=
		\left[
		\begin{array}{ccc|c}
		1 & 0 & \frac{7}{3} & \frac{5}{3} \\
		0 & 1 & \frac{1}{3} & \frac{-1}{3}\\
		0 & 0 & 0 & -3\\
		\end{array}
		\right]
	\]
\begin{problem}{4}
	Solving two planes
\end{problem}
	\[3x_1 + 2x_2 -4x_3 = 1\]
	\[x_1 + x_2 - x_3 = 1\]
	\[x_1 = 1 - x_2 + x_3\]
	
	
	\[3(1 - x_2 + x_3) + 2x_2 - 4x_3 = 1\]
	\[3 - 3x_2 + 3x_3 + 2x_2 - 4x_3 = 1\]
	\[x_2 -x_3 = -2\]
	
	
	\[\frac{-x_2 -x_3 = -2}{-1}\]
	\[x_2 = 2 - x_3\]
	
	
	\[x_1 = 1 - (2 - x_3) + x_3\]
	\[x_1 = 1 - 2 + x_3 + x_3\]
	\[x_1 = -1 + 2x_3\]
	
	\[\vec x = (-1 + 2x_3) + (2 - x_3)\]
	\[\vec x = x_3 + 1\]
The solution set represents a line.


\begin{problem}{5}
	Solving System of Equations/Linear Independance
\end{problem}
	\[\sqbrl\begin{matrix}
		a_{1_1} & a_{2_1} & a_{3_1}\\
		a_{1_2} & a_{2_2} & a_{3_2}\\
		a_{1_3} & a_{2_3} & a_{3_3}\\
	\end{matrix}\sqbrr
	= \sqbrl\begin{matrix}
		2 & 2 & 2\\
		-3 & 6 & -12\\
		4 & 1 & 1\\
		\end{matrix}\sqbrr		
	\]
The System of Equations' coefficients map to the matrix above. This is the same as expanding the equation of 
$x_1 \vec{a}_1 + x_2 \vec{a}_2 + x_3 \vec{a}_3$.\\
Therefore the vectors $\vec a_1,  \vec a_2,  \vec a_3$ are equivalent to the systems of equations.\\
	\[=
		\left[
		\begin{array}{ccc|c}
		2 & 2 & -2 & 0\\
		-3 & 6 & -12 & 0\\
		4 & 1 & 1 & 0\\
		\end{array}
		\right] \frac{1}{2} R_1
	\]
	\[\sim
		\left[
		\begin{array}{ccc|c}
		1 & 1 & -1 & 0\\
		-3 & 6 & -12 & 0\\
		4 & 1 & 1 & 0\\
		\end{array}
		\right] R_2 + 3R_1
	\]
	\[\sim
		\left[
		\begin{array}{ccc|c}
		1 & 1 & -1 & 0\\
		0 & 9 & -15 & 0\\
		4 & 1 & 1 & 0\\
		\end{array}
		\right] R3_ - 4R_1
	\]
	\[\sim
		\left[
		\begin{array}{ccc|c}
		1 & 1 & -1 & 0\\
		0 & 9 & -15 & 0\\
		0 & -3 & 5 & 0\\
		\end{array}
		\right] \frac{1}{9} R_2
	\]
	\[\sim
		\left[
		\begin{array}{ccc|c}
		1 & 1 & -1 & 0\\
		0 & 1 & \frac{-5}{3} & 0\\
		0 & -3 & 5 & 0\\
		\end{array}
		\right] R_3 + 3R_2
	\]
	\[\sim
		\left[
		\begin{array}{ccc|c}
		1 & 1 & -1 & 0\\
		0 & 1 & \frac{-5}{3} & 0\\
		0 & 0 & 0 & 0\\
		\end{array}
		\right] R_3 + 3R_2
	\]
	\[\sim
		\left[
		\begin{array}{ccc|c}
		1 & 1 & -1 & 0\\
		0 & 1 & \frac{-5}{3} & 0\\
		0 & 0 & 0 & 0\\
		\end{array}
		\right] R_1 - R_2
	\]
	\[\sim
		\left[
		\begin{array}{ccc|c}
		1 & 0 & \frac{2}{3} & 0\\
		0 & 1 &  \frac{-5}{3} & 0\\
		0 & 0 & 0 & 0\\
		\end{array}
		\right]
	\]
The system of equations is linearly independant. 

\begin{problem}{6}
	Solving System of Equations/Spanning
\end{problem}
	\[\sqbrl\begin{matrix}
		a_{1_1} & a_{2_1} & a_{3_1}\\
		a_{1_2} & a_{2_2} & a_{3_2}\\
		a_{1_3} & a_{2_3} & a_{3_3}\\
	\end{matrix}\sqbrr
	= \sqbrl\begin{matrix}
		1 & 1 & 0\\
		0 & 1 & 1\\
		0 & 0 & 1\\
		\end{matrix}\sqbrr		
	\]
The System of Equations' coefficients map to the matrix above. This is the same as expanding the equation of 
$x_1 \vec{a}_1 + x_2 \vec{a}_2 + x_3 \vec{a}_3$.\\
Therefore the vectors $\vec a_1,  \vec a_2,  \vec a_3$ are equivalent to the systems of equations.\\
	\[=
		\left[
		\begin{array}{ccc|c}
		1 & 1 & 0 & v_1\\
		0 & 1 & 1 & v_2\\
		0 & 0 & 1 & v_3\\
		\end{array}
		\right] R_1 - R_2
	\]
	\[\sim
		\left[
		\begin{array}{ccc|c}
		1 & 0 & -1 & v_1 - v_2\\
		0 & 1 & 1 & v_2\\
		0 & 0 & 1 & v_3\\
		\end{array}
		\right] R_1 + R_3
	\]
	\[\sim
		\left[
		\begin{array}{ccc|c}
		1 & 0 & 0 & v_1 - v_2 + v_3\\
		0 & 1 & 1 & v_2\\
		0 & 0 & 1 & v_3\\
		\end{array}
		\right] R_2 - R_3
	\]
	\[\sim
		\left[
		\begin{array}{ccc|c}
		1 & 0 & 0 & v_1 - v_2 + v_3\\
		0 & 1 & 0 & v_2 - v_3\\
		0 & 0 & 1 & v_3\\
		\end{array}
		\right]
	\]
System of Equations is consistent. Therefore the vectors $\vec a_1,  \vec a_2,  \vec a_3$ span $\mathds{R}^3$.

\begin{problem}{7}
	Textbooks
\end{problem}
	\[1400 x_b + 1800 x_c + 600 x_p = 658,000\]
	\[1300 x_b + 1700 x_c + 500 x_p = 587,000\]
	\[1200 x_b + 1600 x_c + 400 x_p = 520,000\]
	
	\[=
		\left[
		\begin{array}{ccc|c}
		1400 & 1800 & 600 & 658000\\
		1300 & 1700 & 500 & 587000\\
		1200 & 1600 & 400 & 520000\\
		\end{array}
		\right] \frac{1}{100} \begin{matrix}R_1\\R_2\\R_3\\\end{matrix}
	\]
	\[\sim
		\left[
		\begin{array}{ccc|c}
		14 & 18 & 6 & 6580\\
		13 & 17 & 5 & 5870\\
		12 & 16 & 4 & 5200\\
		\end{array}
		\right] \begin{matrix} \frac{1}{14} R_1\\\\\ \end{matrix}
	\]
	\[\sim
		\left[
		\begin{array}{ccc|c}
		1 & \frac{9}{7} & \frac{3}{7} & 470\\
		13 & 17 & 5 & 5870\\
		12 & 16 & 4 & 5200\\
		\end{array}
		\right] R_2 - 13R_1
	\]
	\[\sim
		\left[
		\begin{array}{ccc|c}
		1 & \frac{9}{7} & \frac{3}{7} & 470\\
		0 & \frac{2}{7} & \frac{-4}{7} & -240\\
		12 & 16 & 4 & 5200\\
		\end{array}
		\right]  R_3 - 12R_1
	\]
	\[\sim
		\left[
		\begin{array}{ccc|c}
		1 & \frac{9}{7} & \frac{3}{7} & 470\\
		0 & \frac{2}{7} & \frac{-4}{7} & -240\\
		0 & \frac{4}{7} & \frac{-8}{7} & -440\\
		\end{array}
		\right] \frac{7}{2}R_2
	\]
	\[\sim
		\left[
		\begin{array}{ccc|c}
		1 & \frac{9}{7} & \frac{3}{7} & 470\\
		0 & 1 & -2 & -840\\
		0 & \frac{4}{7} & \frac{-8}{7} & -440\\
		\end{array}
		\right] R_3 + \frac{-4}{7}R_2
	\]
	\[\sim
		\left[
		\begin{array}{ccc|c}
		1 & \frac{9}{7} & \frac{3}{7} & 470\\
		0 & 1 & -2 & -840\\
		0 & 0 & 0 & 40\\
		\end{array}
		\right] R_1 + \frac{-9}{7}R_2
	\]
	\[\sim
		\left[
		\begin{array}{ccc|c}
		1 & 0 & 3 & 1550\\
		0 & 1 & -2 & -840\\
		0 & 0 & 0 & 40\\
		\end{array}
		\right] 
	\]
Because the RREF is inconsistent, there is no solution. This means the prices of each textbook have not changed as the equations are parallel to eachother. This could have been resolved from the coefficients of the equations as they are multiples of eachother.
\begin{problem}{8}
	PIRATE - YARR
\end{problem}
Putting the System of Equations into an augmented coefficient matrix.
	\[=
		\left[
			\begin{array}{cccc|c}
			1 & 1 & 1 & 1 & 70\\
			0 & 1 & 1 & 0 & 30\\
			0 & 0 & 2 & 2 & 90\\
			0 & 4 & 0 & 1 & 80\\
			\end{array}
		\right] \frac{1}{2} R_3
	\]
	\[\sim
		\left[
			\begin{array}{cccc|c}
			1 & 1 & 1 & 1 & 70\\
			0 & 1 & 1 & 0 & 30\\
			0 & 0 & 1 & 1 & 45\\
			0 & 4 & 0 & 1 & 80\\
			\end{array}
		\right] R_1 - R_2
	\]
	
	\[\sim
		\left[
			\begin{array}{cccc|c}
			1 & 0 & 0 & 1 & 40\\
			0 & 1 & 1 & 0 & 30\\
			0 & 0 & 1 & 1 & 45\\
			0 & 0 & -4 & 1 & -40\\
			\end{array}
		\right] R_4 + 4R_3
	\]
	\[\sim
		\left[
			\begin{array}{cccc|c}
			1 & 0 & 0 & 1 & 40\\
			0 & 1 & 1 & 0 & 30\\
			0 & 0 & 1 & 1 & 45\\
			0 & 0 & 0 & 5 & 140\\
			\end{array}
		\right] \frac{1}{5} R_4
	\]
	\[\sim
		\left[
			\begin{array}{cccc|c}
			1 & 0 & 0 & 1 & 40\\
			0 & 1 & 1 & 0 & 30\\
			0 & 0 & 1 & 1 & 45\\
			0 & 0 & 0 & 1 & 28\\
			\end{array}
		\right] R_3 - R_4
	\]
	\[\sim
		\left[
			\begin{array}{cccc|c}
			1 & 0 & 0 & 1 & 40\\
			0 & 1 & 1 & 0 & 30\\
			0 & 0 & 1 & 0 & 17\\
			0 & 0 & 0 & 1 & 28\\
			\end{array}
		\right] R_2 - R_3
	\]
	\[\sim
		\left[
			\begin{array}{cccc|c}
			1 & 0 & 0 & 1 & 40\\
			0 & 1 & 0 & 0 & 13\\
			0 & 0 & 1 & 0 & 17\\
			0 & 0 & 0 & 1 & 28\\
			\end{array}
		\right] R_1 - R_4
	\]
	\[\sim
		\left[
			\begin{array}{cccc|c}
			1 & 0 & 0 & 0 & 12\\
			0 & 1 & 0 & 0 & 13\\
			0 & 0 & 1 & 0 & 17\\
			0 & 0 & 0 & 1 & 28\\
			\end{array}
		\right]
	\]
\[A = 12, B = 13, C = 17, D = 28\]
\[12 + 13 + 17 + 28 = 70\]
\[13 + 17 = 30\]
\[2(13) + 2(28) = 90\]
\[4(13) + 28 = 80\]

\end{document}
