\documentclass[12pt]{article}
\usepackage[margin=1in]{geometry}
\usepackage{amsthm,amssymb,amsfonts}
\usepackage{tipa}
\usepackage{ gensymb }
\usepackage[fleqn]{amsmath}
\usepackage{dsfont}
 
\newcommand{\N}{\mathbb{N}}
\newcommand{\Z}{\mathbb{Z}}
\newcommand{\dbarl}{\left\lVert}
\newcommand{\dbarr}{\right\rVert}
\newcommand{\sqbrl}{\left[}
\newcommand{\sqbrr}{\right]}

\newenvironment{problem}[2][Problem]{\begin{trivlist}
\item[\hskip \labelsep {\bfseries #1}\hskip \labelsep {\bfseries #2.}]}{\end{trivlist}}
%If you want to title your bold things something different just make another thing exactly like this but replace "problem" with the name of the thing you want, like theorem or lemma or whatever
 
\begin{document}
 
%\renewcommand{\qedsymbol}{\filledbox}
%Good resources for looking up how to do stuff:
%Binary operators: http://www.access2science.com/latex/Binary.html
%General help: http://en.wikibooks.org/wiki/LaTeX/Mathematics
%Or just google stuff
 
\title{MATH 114 - Fall 2016 - Assignment 4}
\author{James Sinn - 20654551}
\maketitle
 
 \begin{problem}{1}
 	Matrix-Vector Products
 \end{problem}
 
a) 2x2 $\cdot$ 2x1 = 2x1\\ 
	 \[\sqbrl\begin{matrix}2 & -4\\5 & -3\\ \end{matrix}\sqbrr \sqbrl\begin{matrix}4\\-2\\\end{matrix}\sqbrr
	 = \sqbrl\begin{matrix}(2\cdot4) + (-4\cdot-2)\\(5\cdot4)+(-3\cdot-2)\\\end{matrix}\sqbrr 
	 = \sqbrl\begin{matrix}16\\26\\\end{matrix}\sqbrr\]
	  
b) 2x3 $\cdot$ 3x1 = 2x1\\
	\[\sqbrl\begin{matrix}1 & 2 & 3\\-3 &-2 & -1\\\end{matrix}\sqbrr\sqbrl\begin{matrix}1\\3\\-1\\\end{matrix}\sqbrr
	= \sqbrl\begin{matrix}4\\-8\\\end{matrix}\sqbrr\]

\begin{problem}{2}
	Matrix-Matrix Products
\end{problem}

a) 2x2 $\cdot$ 2x3 = 2x3\\
	\[\sqbrl\begin{matrix}2 & 2\\3 & 3\\\end{matrix}\sqbrr\sqbrl\begin{matrix}-3 & 4 & 2\\1 & -4 & -6\\\end{matrix}\sqbrr
	=\sqbrl\begin{matrix}-6 + 2 & 8 + -8 & 4 + -12\\-9 + 3 & 12 - 12 & 6 -18\\\end{matrix}\sqbrr
	=\sqbrl\begin{matrix}-4 & 0 & -8\\-6 & 0 & -12\\\end{matrix}\sqbrr\]
 
b) 3x2 $\cdot$ 2x2 = 3x2\\
	\[\sqbrl\begin{matrix}1 & 1\\2 & 2\\3 & 3\\\end{matrix}\sqbrr\sqbrl\begin{matrix}3 & -1\\4 & 5\\\end{matrix}\sqbrr
	=\sqbrl\begin{matrix}3 + 4 & -1 + 5\\6+8 & -2 + 10\\ 9 + 12 & -3 + 15\\\end{matrix}\sqbrr
	=\sqbrl\begin{matrix}7 & 4\\14 & 8\\21 & 12\\\end{matrix}\sqbrr\]
 
 \begin{problem}{3}
 	Matrix Multiplications
 \end{problem}
 
a)\\
 	\[\sqbrl\begin{matrix}1 & 2 & 3\\\end{matrix}\sqbrr\sqbrl\begin{matrix}1\\2\\3\\\end{matrix}\sqbrr
 	=\sqbrl\begin{matrix}1 + 4+ 9\end{matrix}\sqbrr
 	=\sqbrl\begin{matrix}14\end{matrix}\sqbrr\]
 	
b)\\
	\[\sqbrl\begin{matrix}1\\2\\3\\\end{matrix}\sqbrr\sqbrl\begin{matrix}1 & 2 & 3\\\end{matrix}\sqbrr
	=\sqbrl\begin{matrix}1 & 2 & 3\\2 & 4 & 6\\3 & 6 & 9\\\end{matrix}\sqbrr\]
	
\begin{problem}{4}
	Geometric Transformations
\end{problem}
By rotating the result by the inverse of $R_\theta$ we will reverse the rotation made to get $(-1, \sqrt{2}, 0)$.\\
3x3 $\cdot$ 3x1 = 3x1
	\[\sqbrl\begin{matrix}\cos{\frac{\pi}{4}} & \sin{\frac{\pi}{4}}\\-\sin{\frac{\pi}{4}} & \cos{\frac{\pi}{4}}\end{matrix}\sqbrr\sqbrl\begin{matrix}-1\\\sqrt{2}\\\end{matrix}\sqbrr
	= \sqbrl\begin{matrix}\frac{-\sqrt{2}-2}{2}\\\frac{-\sqrt{2}+2}{2}\end{matrix}\sqbrr\]
	
\begin{problem}{5}
	Geometric Transformations
\end{problem}
For all questions below, I assumed that the matrix $A$ was to be multiplied with the input vector $\vec v$. Thus A is the matrix that will change $\vec v$ to get the desired transformation.\\

a)\\
	\[\sqbrl\begin{matrix}\frac{1}{2} & 0\\0 & \frac{1}{2}\end{matrix}\sqbrr\]
	
b)\\ This is a simplification of having $\pi$ as $\theta$ in the rotation vector in $R_{\theta}$\\
	\[\sqbrl\begin{matrix}-1 & 0\\0 & -1\end{matrix}\sqbrr\]
	
c)\\
	\[\sqbrl\begin{matrix}0 & 1\\1 & 0\end{matrix}\sqbrr\]
	
d)\\
	\[\sqbrl\begin{matrix}1 & 0\\0 & 0\end{matrix}\sqbrr\]

\begin{problem}{6}
	Geometric Transformations
\end{problem}
By rotating the vectors $\vec v_L$ and $\vec v_A$ by $\frac{-\pi}{2}$ about the z axis we will acheive the rotation relative to us 6 hours later.\\

$\vec v_L$
\[\sqbrl\begin{matrix}
\cos{\frac{-\pi}{2}} & -\sin{\frac{-\pi}{2}} & 0\\
\sin{\frac{-\pi}{2}} & \cos{\frac{-\pi}{2}} & 0 \\
0 & 0 & 1\\\end{matrix}\sqbrr
\sqbrl\begin{matrix}\frac{1}{\sqrt2}\\0\\\frac{1}{\sqrt2}\end{matrix}\sqbrr
= \sqbrl\begin{matrix}0\\\frac{-\sqrt{2}}{2}\\\frac{\sqrt{2}}{2}\end{matrix}\sqbrr\]


$\vec v_A$
\[\sqbrl\begin{matrix}
\cos{\frac{-\pi}{2}} & -\sin{\frac{-\pi}{2}} & 0\\
\sin{\frac{-\pi}{2}} & \cos{\frac{-\pi}{2}} & 0 \\
0 & 0 & 1\\\end{matrix}\sqbrr
\sqbrl\begin{matrix}\frac{1}{\sqrt8}\\
\frac{\sqrt{3}}{2}\\
\frac{1}{\sqrt{8}}\end{matrix}\sqbrr
= \sqbrl\begin{matrix}
\frac{\sqrt{3}}{2}\\
\frac{-\sqrt{2}}{4}\\
\frac{\sqrt{2}}{4}\end{matrix}\sqbrr\]

The angle between the two vectors initially was $\frac{\pi}{3}$. After the 6 hours had passed, the angle between was $\frac{\pi}{3}$. Thus no change had been made.

\begin{problem}{7}
	Geometric Transformations
\end{problem}

a)\\ 
The set of the columns is $\left\{\sqbrl\begin{matrix}\cos{\theta}\\ \sin{\theta}\\\end{matrix}\sqbrr,\sqbrl\begin{matrix}-\sin{\theta}\\\cos{\theta}\end{matrix}\sqbrr\right\}$ and by substituting any value for $\theta$ that is within the range of 0 to $2\pi$ will produce the scale that we can then reduce down to RREF. To simplify it, I'll use $2\pi$ so I don't have to make it into RREF.\\
	\[S= \left\{\sqbrl\begin{matrix}
	\cos{2\pi}\\
	\sin{2\pi}\\\end{matrix}\sqbrr,
	\sqbrl\begin{matrix}
	-\sin{2\pi}\\
	\cos{2\pi}\end{matrix}\sqbrr
	\right\}
	=
	\left\{\sqbrl\begin{matrix}
	1\\
	0\\\end{matrix}\sqbrr,
	\sqbrl\begin{matrix}
	0\\
	1\end{matrix}\sqbrr
	\right\}
	\]\\
The set $S$ is a basis for $ \rm I\!R^2 $ because it is linearly independant, thus they are orthagonol.\\

b)\\
Continuing from the previous question, and using the same result, the magnitude of each column is 1. \\
	\[S_1 = \sqbrl\begin{matrix}1\\0\end{matrix}\sqbrr\]
	\[\dbarl S_1\dbarr = \sqrt{1^2 + 0^2} = 1\]
	\[S_2 = \sqbrl\begin{matrix}0\\1\end{matrix}\sqbrr\]
	\[\dbarl S_1\dbarr = \sqrt{0^2 + 1^2} = 1\]
	
c)\\
	\[R_{\theta}^T = \sqbrl\begin{matrix}\cos{\theta} & \sin{\theta}\\-\sin{\theta} & \cos{\theta}\end{matrix}\sqbrr\]
	\[R_{\theta}^T\cdot{R_{\theta}}  
	= \sqbrl\begin{matrix}\cos{\theta} & \sin{\theta}\\-\sin{\theta} & \cos{\theta}\end{matrix}\sqbrr\sqbrl\begin{matrix}	\cos{\theta} & -\sin{\theta}\\\sin{\theta} & \cos{\theta}\end{matrix}\sqbrr
	= \sqbrl\begin{matrix}\cos{\theta}^2 + \sin{\theta}^2 & -\sin{\theta}\cos{\theta} + \cos{\theta}\sin{\theta}\\
		-\sin{\theta}\cos{\theta} + \cos{\theta}\sin{\theta} & \sin{\theta}^2 + \cos{\theta}^2\end{matrix}\sqbrr\]
	\[=\sqbrl\begin{matrix}1 & 0\\ 0 & 1\end{matrix}\sqbrr\]

\begin{problem}{8}
	Geometric Transformations
\end{problem}
	\[Q = \sqbrl\begin{matrix}4 & -8 & 4\\
							-2 & 4 & -2\\
							3 & -6 & 3\\
			     \end{matrix}\sqbrr\begin{matrix}R_1 - \frac{1}{3}R_3\\\\\end{matrix}\]
			     
a)\\
	\[\sim
		\sqbrl\begin{matrix}
		3 & -6 & 3\\
		-2 & 4 & -2\\
		3 & -6 & 3\\
		\end{matrix}\sqbrr\begin{matrix}\\\\R_3-R_1\end{matrix}\]
	\[\sim
		\sqbrl\begin{matrix}
		3 & -6 & 3\\
		-2 & 4 & -2\\
		0 & 0 & 0\\
		\end{matrix}\sqbrr\begin{matrix}\frac{1}{3}R_1\\\\\frac{1}{2}R_2\\\end{matrix}\]
	\[\sim
		\sqbrl\begin{matrix}
		1 & -2 & 1\\
		-1 & 2 & -1\\
		0 & 0 & 0\\
		\end{matrix}\sqbrr\begin{matrix}\\R_2+R_1\\\end{matrix}\]
	\[\sim
		\sqbrl\begin{matrix}
		1 & -2 & 1\\
		0 & 0 & 0\\
		0 & 0 & 0\\
		\end{matrix}\sqbrr\]
This represents a plane. There are two free variables. This is because all three equations are the same plane, thus they intersect in infinitely many points.
\[x_1 -2x_2 +x_3 = 0\]

b)\\
	\[Q\vec x = \sqbrl\begin{matrix}
		1 & -2 & 1\\
		0 & 0 & 0\\
		0 & 0 & 0\\
		\end{matrix}\sqbrr\sqbrl\begin{matrix}1\\-2\\1\end{matrix}\sqbrr
		= \sqbrl\begin{matrix}6\\0\\0\end{matrix}\sqbrr\]
	\[Q\vec y = \sqbrl\begin{matrix}
		1 & -2 & 1\\
		0 & 0 & 0\\
		0 & 0 & 0\\
		\end{matrix}\sqbrr\sqbrl\begin{matrix}0\\2\\0\end{matrix}\sqbrr
		= \sqbrl\begin{matrix}-4\\0\\0\end{matrix}\sqbrr\]
	\[Q\vec x = \sqbrl\begin{matrix}
		1 & -2 & 1\\
		0 & 0 & 0\\
		0 & 0 & 0\\
		\end{matrix}\sqbrr\sqbrl\begin{matrix}1\\3\\-1\end{matrix}\sqbrr
		= \sqbrl\begin{matrix}-6\\0\\0\end{matrix}\sqbrr\]
A simple geometric shape that contains these three vectors is a line.\\

c)\\
All three vectors found are scalar multiples of $\sqbrl\begin{matrix}1\\0\\0\end{matrix}\sqbrr$ which is equivelant to the columnspace of $Q$ because the columnspace is also scalar multiples of $\sqbrl\begin{matrix}1\\0\\0\end{matrix}\sqbrr$.
\begin{problem}{9}
	Geometric Transformations
\end{problem}
	\[M = \sqbrl\begin{matrix}1 & 1 & 5\\0 & 1 & 2\\3& -5 & -1\end{matrix}\sqbrr\]
	
a)\\
	\[\sim
	\sqbrl\begin{matrix}
	1 & 1 & 5\\
	0 & 1 & 2\\
	3& -5 & -1\end{matrix}\sqbrr\begin{matrix}\\\\R_3 - 3R_1\end{matrix}\]
	\[\sim
	\sqbrl\begin{matrix}
	1 & 1 & 5\\
	0 & 1 & 2\\
	0& -8 & -16\end{matrix}\sqbrr\begin{matrix}\\\\R_3 + 8R_2\end{matrix}\]
	\[\sim
	\sqbrl\begin{matrix}
	1 & 1 & 5\\
	0 & 1 & 2\\
	0& 0 & 0\end{matrix}\sqbrr\begin{matrix}\\\\R_1- R_2\end{matrix}\]
	\[\sim
	\sqbrl\begin{matrix}
	1 & 0 & 3\\
	0 & 1 & 2\\
	0& 0 & 0\end{matrix}\sqbrr\]
$M$ is of rank 2. With 1 free variable. This means that $M$'s solution set represents a line.\\

b)\\
	\[a_1\vec{m_1} + a_2\vec{m_2} + a_3\vec{m_3} = 0\]
	\[a_1\sqbrl\begin{matrix}1\\0\\0\end{matrix}\sqbrr + a_2\sqbrl\begin{matrix}0\\1\\0\end{matrix}\sqbrr + a_3\sqbrl\begin{matrix}3\\2\\0\end{matrix}\sqbrr = 0\]
	\[-3\sqbrl\begin{matrix}1\\0\\0\end{matrix}\sqbrr -2\sqbrl\begin{matrix}0\\1\\0\end{matrix}\sqbrr + 1\sqbrl\begin{matrix}3\\2\\0\end{matrix}\sqbrr = 0\]
\[a_1 = -3, a_2 = -2, a_3 = 1\] 
Because the linear combination of the columns have a solution where $a_1, a_2, a_3$ are not all 0, the columnspace of $M$ is linearly dependant.\\

c)\\
The nullspace of $M$ is $\left\{\sqbrl\begin{matrix}-3\\-2\\1\end{matrix}\sqbrr\right\}$

d)\\
The columnspace of $M$ is $\left\{\sqbrl\begin{matrix}1\\0\\0\end{matrix}\sqbrr, \sqbrl\begin{matrix}0\\1\\0\end{matrix}\sqbrr\right\}$

e)\\
The sum of dim(null($M$)) and dim(col($M$)) is 3.\\
dim(null($M$)) = 1, and dim(col($M$)) = 2\\
f)\\
The basis for Col($M$) is $\left\{\sqbrl\begin{matrix}1\\0\\0\end{matrix}\sqbrr, \sqbrl\begin{matrix}0\\1\\0\end{matrix}\sqbrr\right\}$
\end{document}
