\documentclass[12pt]{article}
\usepackage[margin=1in]{geometry}
\usepackage{amsthm,amssymb,amsfonts}
\usepackage{tipa}
\usepackage{ gensymb }
\usepackage[fleqn]{amsmath}
\usepackage{dsfont}
 
\newcommand{\N}{\mathbb{N}}
\newcommand{\Z}{\mathbb{Z}}
\newcommand{\dbarl}{\left\lVert}
\newcommand{\dbarr}{\right\rVert}
\newcommand{\sqbrl}{\left[}
\newcommand{\sqbrr}{\right]}

\newenvironment{problem}[2][Problem]{\begin{trivlist}
\item[\hskip \labelsep {\bfseries #1}\hskip \labelsep {\bfseries #2.}]}{\end{trivlist}}
%If you want to title your bold things something different just make another thing exactly like this but replace "problem" with the name of the thing you want, like theorem or lemma or whatever
 
\begin{document}
 
%\renewcommand{\qedsymbol}{\filledbox}
%Good resources for looking up how to do stuff:
%Binary operators: http://www.access2science.com/latex/Binary.html
%General help: http://en.wikibooks.org/wiki/LaTeX/Mathematics
%Or just google stuff
 
\title{MATH 114 - Fall 2016 - Assignment 4}
\author{James Sinn - 20654551}
\maketitle
 
 \begin{problem}{1}
 	Matrix-Vector Products
 \end{problem}
 
a) 2x2 $\cdot$ 2x1 = 2x1\\ 
	 \[\sqbrl\begin{matrix}2 & -4\\5 & -3\\ \end{matrix}\sqbrr \sqbrl\begin{matrix}4\\-2\\\end{matrix}\sqbrr
	 = \sqbrl\begin{matrix}(2\cdot4) + (-4\cdot-2)\\(5\cdot4)+(-3\cdot-2)\\\end{matrix}\sqbrr 
	 = \sqbrl\begin{matrix}16\\26\\\end{matrix}\sqbrr\]
	  
b) 2x3 $\cdot$ 3x1 = 2x1\\
	\[\sqbrl\begin{matrix}1 & 2 & 3\\-3 &-2 & -1\\\end{matrix}\sqbrr\sqbrl\begin{matrix}1\\3\\-1\\\end{matrix}\sqbrr
	= \sqbrl\begin{matrix}4\\-8\\\end{matrix}\sqbrr\]

\begin{problem}{2}
	Matrix-Matrix Products
\end{problem}
a) 2x2 $\cdot$ 2x3 = 2x3\\
	\[\sqbrl\begin{matrix}2 & 2\\3 & 3\\\end{matrix}\sqbrr\sqbrl\begin{matrix}-3 & 4 & 2\\1 & -4 & -6\\\end{matrix}\sqbrr
	=\sqbrl\begin{matrix}-6 + 2 & 8 + -8 & 4 + -12\\-9 + 3 & 12 - 12 & 6 -18\\\end{matrix}\sqbrr
	=\sqbrl\begin{matrix}-4 & 0 & -8\\-6 & 0 & -12\\\end{matrix}\sqbrr\]
 
b) 3x2 $\cdot$ 2x2 = 3x2\\
	\[\sqbrl\begin{matrix}1 & 1\\2 & 2\\3 & 3\\\end{matrix}\sqbrr\sqbrl\begin{matrix}3 & -1\\4 & 5\\\end{matrix}\sqbrr
	=\sqbrl\begin{matrix}3 + 4 & -1 + 5\\6+8 & -2 + 10\\ 9 + 12 & -3 + 15\\\end{matrix}\sqbrr
	=\sqbrl\begin{matrix}7 & 4\\14 & 8\\21 & 12\\\end{matrix}\sqbrr\]
 
 \begin{problem}{3}
 	Matrix Multiplications
 \end{problem}
 a)\\
 	\[\sqbrl\begin{matrix}1 & 2 & 3\\\end{matrix}\sqbrr\sqbrl\begin{matrix}1\\2\\3\\\end{matrix}\sqbrr
 	=\sqbrl\begin{matrix}1 + 4+ 9\end{matrix}\sqbrr
 	=\sqbrl\begin{matrix}14\end{matrix}\sqbrr\]
b)\\
	\[\sqbrl\begin{matrix}1\\2\\3\\\end{matrix}\sqbrr\sqbrl\begin{matrix}1 & 2 & 3\\\end{matrix}\sqbrr
	=\sqbrl\begin{matrix}1 & 2 & 3\\2 & 4 & 6\\3 & 6 & 9\\\end{matrix}\sqbrr\]
	
\begin{problem}{4}
	Geometric Transformations
\end{problem}
By rotating the result by $\frac{-\pi}{4}$ we will reverse the rotation made to get $(-1, \sqrt{2}, 0)$.\\
3x3 $\cdot$ 3x1 = 3x1
	\[\sqbrl\begin{matrix}\cos{\frac{-\pi}{4}} & \sin{\frac{-\pi}{4}}\\-\sin{\frac{-\pi}{4}} & \cos{\frac{-\pi}{4}}\end{matrix}\sqbrr\sqbrl\begin{matrix}-1\\\sqrt{2}\\\end{matrix}\sqbrr
	= \sqbrl\begin{matrix}\frac{-\sqrt{2}-2}{2}\\\frac{-\sqrt{2}+2}{2}\end{matrix}\sqbrr\]
	
\begin{problem}{5}
	Geometric Transformations
\end{problem}
For all questions below, I assumed that the matrix $A$ was to be multiplied with the input vector $\vec v$. Thus A is the matrix that will change $\vec v$ to get the desired transformation.\\
a)\\
	\[\sqbrl\begin{matrix}\frac{1}{2} & 0\\0 & \frac{1}{2}\end{matrix}\sqbrr\]
b) - This is a simplification of having $\pi$ as $\theta$ in the rotation vector.\\
	\[\sqbrl\begin{matrix}-1 & 0\\0 & -1\end{matrix}\sqbrr\]
c) - \\

d) \\

\begin{problem}{6}
	Geometric Transformations
\end{problem}

\begin{problem}{7}
	Geometric Transformations
\end{problem}
a)\\
\end{document}
